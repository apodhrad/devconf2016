\documentclass{beamer}
\usetheme{AnnArbor}
\usecolortheme{beaver}
\usepackage{tikz}
\usepackage{color}
\usepackage{listings}

\lstset{language=Java,
  basicstyle=\footnotesize\ttfamily,
  keywordstyle=\footnotesize\color{blue}\ttfamily,
  commentstyle=\footnotesize\color{gray}\ttfamily,
}

\definecolor{darkred}{rgb}{0.8,0,0}

\setbeamercolor{title}{fg=white,bg=darkred!80!black}
\setbeamercolor{frametitle}{fg=darkred!80!black,bg=white}
%\setbeamercolor{section in head/foot}{fg=green,bg=yellow}
%\setbeamercolor{subsection in head/foot}{bg=white}
\begin{document}
\title{Advanced Java Debugging}   
\author{Andrej Podhradsky}
\date{\today} 
%\logo{\includegraphics[height=1cm]{reddeer_logo.png}\vspace{220pt}}

\addtobeamertemplate{title page}{\center{\includegraphics[height=1cm]{devconf_logo.png}}}{}

\addtobeamertemplate{frametitle}{}{
\begin{tikzpicture}[remember picture,overlay]
\node[anchor=north east,yshift=-8pt] at (current page.north east) {\includegraphics[height=1cm]{devconf_logo.png}};
\end{tikzpicture}}

\frame{\titlepage} 

\frame{\frametitle{Table of contents}\tableofcontents} 

\section{Motivation}

\subsection{For who and why}
\begin{frame}[fragile]
\frametitle{For who and why}
\begin{itemize}
\item for beginners, students and anyone with the lack of debugging experiences
\item debugging is underestimated
\item why is debugging important ...
\end{itemize}
\end{frame}

\subsection{Example}
\begin{frame}[fragile]
\frametitle{How would you debug this?}
\begin{lstlisting}
for (c = 0; c < ( n - 1 ); c++) {
  for (d = 0; d < n - c - 1; d++) {
    if (array[d] > array[d+1]) {
      swap       = array[d];
      array[d]   = array[d+1];
      array[d+1] = swap;
    }
  }
}
\end{lstlisting}
\end{frame}


\subsection{First debug}
\begin{frame}[fragile]
\frametitle{Using sysout}
\begin{lstlisting}
for (c = 0; c < ( n - 1 ); c++) {
  for (d = 0; d < n - c - 1; d++) {
    if (array[d] > array[d+1]) {
      swap       = array[d];
      System.out.println("LOG swap = " + swap);
      array[d]   = array[d+1];
      System.out.println("LOG array[d] = " + array[d]);
      array[d+1] = swap;
      System.out.println("LOG array[d+1] = " + array[d+1]);
    }
  }
}
\end{lstlisting}
\end{frame}

\subsection{MyCoolApp}
\begin{frame}[fragile]
\frametitle{MyCoolApp}
\begin{itemize}
\item In this workshop we will use MyCoolApp
\item This app is ...
\item and it contains lots of bugs
\item We will dubug it using Eclipse IDE
\item ... and try to fix it
\end{itemize}
\end{frame}

\section{Prepare your environment}

\subsection{Prepare your environment}
\begin{frame}[fragile]
\frametitle{Prepare your environment}
\begin{itemize}
\item JDK 6+
  \begin{lstlisting}
    java -version
  \end{lstlisting}
\item Maven 3.0.5+
  \begin{lstlisting}
    mvn -v
  \end{lstlisting}
\item Eclipse IDE for Java (or JavaEE) Developers
\item Import MyCoolAp as a maven project and run tests
\end{itemize}
\end{frame}

\section{Debugging}

\subsection{Breakpoints and variables}
\begin{frame}[fragile]
\frametitle{Breakpoints and variables}
\begin{itemize}
\item in a code
\item on an attribute
\item on a method
\item on an exception
\item view of variables
\end{itemize}
\end{frame}


\subsection{Conditional breakpoints}
\begin{frame}[fragile]
\frametitle{Conditional breakpoints}
\end{frame}

\subsection{Breakpoints and variables}
\begin{frame}[fragile]
\frametitle{Breakpoints and variables}
\end{frame}

\subsection{Display and watch the code}
\begin{frame}[fragile]
\frametitle{Display and watch the code}
\end{frame}

\subsection{Repeat the code}
\begin{frame}[fragile]
\frametitle{Repeat the code}
\end{frame}

\subsection{Remote debugging}
\begin{frame}[fragile]
\frametitle{Remote debugging}
\end{frame}

\section{Learning Materials}
\begin{frame}[fragile]
\frametitle{Learning Materials}
\begin{itemize}
\item Effective Java Debugging with Eclipse\\\url{http://eclipsesource.com/blogs/2013/01/08/effective-java-debugging-with-eclipse}
\item Java Debugging with Eclipse - Tutorial\\url{http://www.vogella.com/tutorials/EclipseDebugging/article.html}
\end{itemize}
\end{frame}

\end{document}
