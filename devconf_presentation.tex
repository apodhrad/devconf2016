\documentclass{beamer}
\usetheme{AnnArbor}
\usecolortheme{beaver}
\usepackage{tikz}
\usepackage{color}
\usepackage{listings}

\lstset{language=Java,
  basicstyle=\footnotesize\ttfamily,
  keywordstyle=\footnotesize\color{blue}\ttfamily,
  commentstyle=\footnotesize\color{gray}\ttfamily,
}

\definecolor{darkred}{rgb}{0.8,0,0}

\setbeamercolor{title}{fg=white,bg=darkred!80!black}
\setbeamercolor{frametitle}{fg=darkred!80!black,bg=white}
%\setbeamercolor{section in head/foot}{fg=green,bg=yellow}
%\setbeamercolor{subsection in head/foot}{bg=white}
\begin{document}
\title{Advanced Java Debugging}   
\author{Andrej Podhradsky}
\date{\today} 
%\logo{\includegraphics[height=1cm]{reddeer_logo.png}\vspace{220pt}}

\addtobeamertemplate{title page}{\center{\includegraphics[height=1cm]{devconf_logo.png}}}{}

\addtobeamertemplate{frametitle}{}{
\begin{tikzpicture}[remember picture,overlay]
\node[anchor=north east,yshift=-8pt] at (current page.north east) {\includegraphics[height=1cm]{devconf_logo.png}};
\end{tikzpicture}}

\frame{\titlepage} 

\frame{\frametitle{Table of contents}\tableofcontents} 

\section{Introduction}
\subsection{For who and why}
\begin{frame}[fragile]
\frametitle{For who and why}
\begin{itemize}
\item for beginners, students and anyone with a lack od debugging experiences
\item debugging is underestimated
\begin{lstlisting}
new PushButton("OK").click();
new LabeledText("Name").setText("test-name");
new TextEditor("Demo.java").save();
\end{lstlisting}
\item How it works?
\begin{itemize}
\item Each component is found in the constructor
\item The appropriate method (or event) is called in the main thread
\item Waiting until the operation has finished
\end{itemize}
\end{itemize}
\end{frame}

\subsection{Example - part 1}
\begin{frame}[fragile]
\frametitle{How would you debug this?}
\begin{lstlisting}
for (c = 0; c < ( n - 1 ); c++) {
  for (d = 0; d < n - c - 1; d++) {
    if (array[d] > array[d+1]) {
      swap       = array[d];
      array[d]   = array[d+1];
      array[d+1] = swap;
    }
  }
}
\end{lstlisting}
\end{frame}


\subsection{Example - part 2}
\begin{frame}[fragile]
\frametitle{Using sysout}
\begin{lstlisting}
for (c = 0; c < ( n - 1 ); c++) {
  for (d = 0; d < n - c - 1; d++) {
    if (array[d] > array[d+1]) {
      swap       = array[d];
      System.out.println("LOG swap = " + swap);
      array[d]   = array[d+1];
      System.out.println("LOG array[d] = " + array[d]);
      array[d+1] = swap;
      System.out.println("LOG array[d+1] = " + array[d+1]);
    }
  }
}
\end{lstlisting}
\end{frame}

\section{Test Execution}

\subsection{Prerequisite}
\begin{frame}[fragile]
\frametitle{Prerequisite}
\begin{itemize}
\item Download RedDeer IDE 0.6.0\\\url{http://download.eng.brq.redhat.com/scratch/apodhrad}
\item Checkout RedDeer quickstart\\\url{https://github.com/apodhrad/reddeer-quickstart}
\item Display server, e.g. vnc\\We need isolated display since other events can affect our tests
\begin{lstlisting}[language=sh]
sudo yum install tigervnc tigervnc-server
vncpasswd    // set password
vncserver :2 -geometry 1500x900
vncviewer :2
\end{lstlisting}
\item Run the quickstart
\begin{lstlisting}[language=sh]
DISPLAY=:2 mvn clean verify
\end{lstlisting}
\item Each test class must be annotated with
\begin{lstlisting}[language=Java]
@RunWith(RedDeerSuite.class)
\end{lstlisting}
\end{itemize}
\end{frame}

\subsection{Debugging}
\begin{frame}[fragile]
\frametitle{Debugging}
\begin{itemize}
\item You can debug RedDeer tests as any other JUnit tests
\item Remote debugging (on localhost, jenkins, ...)
\begin{lstlisting}[language=sh]
DISPLAY=:2 mvn clean verify -DdebugPort=8001
\end{lstlisting}
Then, in IDE do as follows
\begin{enumerate}
\item Select Run \textgreater Debug Configurations...
\item Create new Remote Java Application
\item Select Standard (Socket Attach) as connection type
\item Set package, host and port (8001)
\end{enumerate}
\end{itemize}
\end{frame}



\section{Learning Materials}
\begin{frame}[fragile]
\frametitle{Learning Materials}
\begin{itemize}
\item Effective Java Debugging with Eclipse\\\url{http://eclipsesource.com/blogs/2013/01/08/effective-java-debugging-with-eclipse}
\item Java Debugging with Eclipse - Tutorial\\url{http://www.vogella.com/tutorials/EclipseDebugging/article.html}
\end{itemize}
\end{frame}

\end{document}
